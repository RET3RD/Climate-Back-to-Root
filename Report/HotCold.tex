\section*{Hottest and Coldest.}
First, it has to be determined which day is the hottest and coldest day, respectively, for a given year.\\
For the hottest day, this was done with the C++ function \texttt{max\_element}, which returns the pointer to the largest element of a certain range of addresses (one row of the temperature array in this case), and the function \texttt{distance}, which returns the difference between two addresses as an integer.\\
The determination of the coldest day had to be written in a more manual way since the set value of $-275$ used to flag datapoints as discarded would always be found as the lowest element of the row. The solution was to write a function that consists of a \texttt{for} loop over all days of the year. In this way, flagged measures can be disregarded as usual via an \texttt{if} clause. For all other valid days, a temporary variable will store the coldest day until a given time of the loop. The temperature of this day will be compared to the temperature of the currently considered day. If the latter is smaller than the former, the former will be overridden by the latter. This makes the temporary variable contain the day of the coldest temperature of a given year at the end of the loop. \par
By these means, one can call the functions just described in a loop that stores the respective result in two arrays for the hottest and coldest days of all given years. These can now be used to fill two histograms that count how often a given day was the hottest or coldest day of the year.%