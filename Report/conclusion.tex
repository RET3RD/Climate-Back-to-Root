\section*{Conclusion}
After extended testing and corrections of the code one of the main and most important observations that were made was that where the data comes from and how the data is given from the stations part is just as important as how it is analyzed.
The first issue with this was encountered when trying to analyze the data from Uppsala. Here the first problem was that this was the only data that had a header before everything with completely useless information to what we where actually looking for. The second problem from this data set was that the data that was listed was both incomplete and inconsistent. As in when we made a closer observation on it what became apparent was that Uppsala had periods in time where they simply did not take any data at all for years as well as that they had both gathered data from other stations as well as extrapolated said data. This led to an utterly unusable and unreadable fit once it was run through the program. The best way we found to circumvent this issue was to simply omit Uppsala's data all together. 

A somewhat similar problem was found with the data from Falun weather station. Where in this case they did not operate at all during Christmas time as in from about 23rd of December all the way to 7th January. This in turn meant that it caused problems when reading the date.
 